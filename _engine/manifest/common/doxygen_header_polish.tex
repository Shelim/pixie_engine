% Latex header for doxygen 1.8.13
\documentclass[twoside]{book}

% Packages required by doxygen
\usepackage{fixltx2e}
\usepackage{calc}
\usepackage{doxygen}
\usepackage[export]{adjustbox} % also loads graphicx
\usepackage{graphicx}
\usepackage[utf8]{inputenc}
\usepackage{makeidx}
\usepackage{multicol}
\usepackage{multirow}
\PassOptionsToPackage{warn}{textcomp}
\usepackage{textcomp}
\usepackage[nointegrals]{wasysym}
\usepackage[table]{xcolor}
\usepackage{longtable}

% NLS support packages
\usepackage{polski}
\usepackage[T1]{fontenc}

% Font selection
\usepackage[T1]{fontenc}
\usepackage[scaled=.90]{helvet}
\usepackage{courier}
\usepackage{amssymb}
\usepackage{sectsty}
\renewcommand{\familydefault}{\sfdefault}
\allsectionsfont{%
  \fontseries{bc}\selectfont%
  \color{darkgray}%
}
\renewcommand{\DoxyLabelFont}{%
  \fontseries{bc}\selectfont%
  \color{darkgray}%
}
\newcommand{\+}{\discretionary{\mbox{\scriptsize$$}}{}{}}

% Page & text layout
\usepackage{geometry}
\geometry{%
  a4paper,%
  top=2.5cm,%
  bottom=2.5cm,%
  left=2.5cm,%
  right=2.5cm%
}
\tolerance=750
\hfuzz=15pt
\hbadness=750
\setlength{\emergencystretch}{15pt}
\setlength{\parindent}{0cm}
\setlength{\parskip}{3ex plus 2ex minus 2ex}
\makeatletter
\renewcommand{\paragraph}{%
  \@startsection{paragraph}{4}{0ex}{-1.0ex}{1.0ex}{%
    \normalfont\normalsize\bfseries\SS@parafont%
  }%
}
\renewcommand{\subparagraph}{%
  \@startsection{subparagraph}{5}{0ex}{-1.0ex}{1.0ex}{%
    \normalfont\normalsize\bfseries\SS@subparafont%
  }%
}
\makeatother

% Headers & footers
\usepackage{fancyhdr}
\pagestyle{fancyplain}
\fancyhead[LE]{\fancyplain{}{\bfseries\thepage}}
\fancyhead[CE]{\fancyplain{}{}}
\fancyhead[RE]{\fancyplain{}{\bfseries\leftmark}}
\fancyhead[LO]{\fancyplain{}{\bfseries\rightmark}}
\fancyhead[CO]{\fancyplain{}{}}
\fancyhead[RO]{\fancyplain{}{\bfseries\thepage}}
\fancyfoot[LE]{\fancyplain{}{}}
\fancyfoot[CE]{\fancyplain{}{}}
\fancyfoot[RE]{\fancyplain{}{\bfseries\scriptsize}}
\fancyfoot[LO]{\fancyplain{}{\bfseries\scriptsize}}
\fancyfoot[CO]{\fancyplain{}{}}
\fancyfoot[RO]{\fancyplain{}{}}
\renewcommand{\footrulewidth}{0.4pt}
\renewcommand{\chaptermark}[1]{%
  \markboth{#1}{}%
}
\renewcommand{\sectionmark}[1]{%
  \markright{\thesection\ #1}%
}

% Indices & bibliography
\usepackage{natbib}
\usepackage[titles]{tocloft}
\setcounter{tocdepth}{3}
\setcounter{secnumdepth}{5}
\makeindex

% Custom commands
\newcommand{\clearemptydoublepage}{%
  \newpage{\pagestyle{empty}\cleardoublepage}%
}

\usepackage{caption}
\captionsetup{labelsep=space,justification=centering,font={bf},singlelinecheck=off,skip=4pt,position=top}

%===== C O N T E N T S =====

\begin{document}

% Titlepage & ToC
\pagenumbering{alph}
\begin{titlepage}


	\centering % Centre everything on the title page
	
	\scshape % Use small caps for all text on the title page
	
	\vspace*{\baselineskip} % White space at the top of the page
	
	%------------------------------------------------
	%	Title
	%------------------------------------------------
	
	\includegraphics[width=0.5\textwidth]{logo_uni.png}
	
	\vspace{4.5\baselineskip} % Whitespace above the title
	
	{\LARGE \textbf {Piotr Kosek} } % Title
	
	\vspace{3.0\baselineskip} % Whitespace below the title
	
	\begin{flushleft}
	Kierunek: informatyka\\
	Specjalność: sieci komputerowe\\
	Numer albumu: 289532
	\end{flushleft}
	
	\vspace{2.0\baselineskip} % Whitespace below the title
	
	{\LARGE Implementacja współczesnego cross-platformowego silnika gier 2D w C++\par } % Title
	
	\vspace{6.5\baselineskip} % Whitespace below the title
	
	\begin{flushright}
	\textbf {Praca inżynierska}\\
	wykonana pod kierunkiem\\
	dr Krzysztofa Podlaskiego\\
	WFiIS UŁ
	\end{flushright}
	
	\vspace{12\baselineskip} % Whitespace after the title block
	
	%------------------------------------------------
	%	Editor(s)
	%------------------------------------------------
	
	Łódź 2018

\end{titlepage}
\pagenumbering{roman}
\tableofcontents
\clearemptydoublepage
\pagenumbering{arabic}

%--- Begin generated contents ---
